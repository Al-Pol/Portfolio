\documentclass[11pt]{article}

    \usepackage[breakable]{tcolorbox}
    \usepackage{parskip} % Stop auto-indenting (to mimic markdown behaviour)
    

    % Basic figure setup, for now with no caption control since it's done
    % automatically by Pandoc (which extracts ![](path) syntax from Markdown).
    \usepackage{graphicx}
    % Maintain compatibility with old templates. Remove in nbconvert 6.0
    \let\Oldincludegraphics\includegraphics
    % Ensure that by default, figures have no caption (until we provide a
    % proper Figure object with a Caption API and a way to capture that
    % in the conversion process - todo).
    \usepackage{caption}
    \DeclareCaptionFormat{nocaption}{}
    \captionsetup{format=nocaption,aboveskip=0pt,belowskip=0pt}

    \usepackage{float}
    \floatplacement{figure}{H} % forces figures to be placed at the correct location
    \usepackage{xcolor} % Allow colors to be defined
    \usepackage{enumerate} % Needed for markdown enumerations to work
    \usepackage{geometry} % Used to adjust the document margins
    \usepackage{amsmath} % Equations
    \usepackage{amssymb} % Equations
    \usepackage{textcomp} % defines textquotesingle
    % Hack from http://tex.stackexchange.com/a/47451/13684:
    \AtBeginDocument{%
        \def\PYZsq{\textquotesingle}% Upright quotes in Pygmentized code
    }
    \usepackage{upquote} % Upright quotes for verbatim code
    \usepackage{eurosym} % defines \euro

    \usepackage{iftex}
    \ifPDFTeX
        \usepackage[T1]{fontenc}
        \IfFileExists{alphabeta.sty}{
              \usepackage{alphabeta}
          }{
              \usepackage[mathletters]{ucs}
              \usepackage[utf8x]{inputenc}
          }
    \else
        \usepackage{fontspec}
        \usepackage{unicode-math}
    \fi

    \usepackage{fancyvrb} % verbatim replacement that allows latex
    \usepackage{grffile} % extends the file name processing of package graphics
                         % to support a larger range
    \makeatletter % fix for old versions of grffile with XeLaTeX
    \@ifpackagelater{grffile}{2019/11/01}
    {
      % Do nothing on new versions
    }
    {
      \def\Gread@@xetex#1{%
        \IfFileExists{"\Gin@base".bb}%
        {\Gread@eps{\Gin@base.bb}}%
        {\Gread@@xetex@aux#1}%
      }
    }
    \makeatother
    \usepackage[Export]{adjustbox} % Used to constrain images to a maximum size
    \adjustboxset{max size={0.9\linewidth}{0.9\paperheight}}

    % The hyperref package gives us a pdf with properly built
    % internal navigation ('pdf bookmarks' for the table of contents,
    % internal cross-reference links, web links for URLs, etc.)
    \usepackage{hyperref}
    % The default LaTeX title has an obnoxious amount of whitespace. By default,
    % titling removes some of it. It also provides customization options.
    \usepackage{titling}
    \usepackage{longtable} % longtable support required by pandoc >1.10
    \usepackage{booktabs}  % table support for pandoc > 1.12.2
    \usepackage{array}     % table support for pandoc >= 2.11.3
    \usepackage{calc}      % table minipage width calculation for pandoc >= 2.11.1
    \usepackage[inline]{enumitem} % IRkernel/repr support (it uses the enumerate* environment)
    \usepackage[normalem]{ulem} % ulem is needed to support strikethroughs (\sout)
                                % normalem makes italics be italics, not underlines
    \usepackage{mathrsfs}
    

    
    % Colors for the hyperref package
    \definecolor{urlcolor}{rgb}{0,.145,.698}
    \definecolor{linkcolor}{rgb}{.71,0.21,0.01}
    \definecolor{citecolor}{rgb}{.12,.54,.11}

    % ANSI colors
    \definecolor{ansi-black}{HTML}{3E424D}
    \definecolor{ansi-black-intense}{HTML}{282C36}
    \definecolor{ansi-red}{HTML}{E75C58}
    \definecolor{ansi-red-intense}{HTML}{B22B31}
    \definecolor{ansi-green}{HTML}{00A250}
    \definecolor{ansi-green-intense}{HTML}{007427}
    \definecolor{ansi-yellow}{HTML}{DDB62B}
    \definecolor{ansi-yellow-intense}{HTML}{B27D12}
    \definecolor{ansi-blue}{HTML}{208FFB}
    \definecolor{ansi-blue-intense}{HTML}{0065CA}
    \definecolor{ansi-magenta}{HTML}{D160C4}
    \definecolor{ansi-magenta-intense}{HTML}{A03196}
    \definecolor{ansi-cyan}{HTML}{60C6C8}
    \definecolor{ansi-cyan-intense}{HTML}{258F8F}
    \definecolor{ansi-white}{HTML}{C5C1B4}
    \definecolor{ansi-white-intense}{HTML}{A1A6B2}
    \definecolor{ansi-default-inverse-fg}{HTML}{FFFFFF}
    \definecolor{ansi-default-inverse-bg}{HTML}{000000}

    % common color for the border for error outputs.
    \definecolor{outerrorbackground}{HTML}{FFDFDF}

    % commands and environments needed by pandoc snippets
    % extracted from the output of `pandoc -s`
    \providecommand{\tightlist}{%
      \setlength{\itemsep}{0pt}\setlength{\parskip}{0pt}}
    \DefineVerbatimEnvironment{Highlighting}{Verbatim}{commandchars=\\\{\}}
    % Add ',fontsize=\small' for more characters per line
    \newenvironment{Shaded}{}{}
    \newcommand{\KeywordTok}[1]{\textcolor[rgb]{0.00,0.44,0.13}{\textbf{{#1}}}}
    \newcommand{\DataTypeTok}[1]{\textcolor[rgb]{0.56,0.13,0.00}{{#1}}}
    \newcommand{\DecValTok}[1]{\textcolor[rgb]{0.25,0.63,0.44}{{#1}}}
    \newcommand{\BaseNTok}[1]{\textcolor[rgb]{0.25,0.63,0.44}{{#1}}}
    \newcommand{\FloatTok}[1]{\textcolor[rgb]{0.25,0.63,0.44}{{#1}}}
    \newcommand{\CharTok}[1]{\textcolor[rgb]{0.25,0.44,0.63}{{#1}}}
    \newcommand{\StringTok}[1]{\textcolor[rgb]{0.25,0.44,0.63}{{#1}}}
    \newcommand{\CommentTok}[1]{\textcolor[rgb]{0.38,0.63,0.69}{\textit{{#1}}}}
    \newcommand{\OtherTok}[1]{\textcolor[rgb]{0.00,0.44,0.13}{{#1}}}
    \newcommand{\AlertTok}[1]{\textcolor[rgb]{1.00,0.00,0.00}{\textbf{{#1}}}}
    \newcommand{\FunctionTok}[1]{\textcolor[rgb]{0.02,0.16,0.49}{{#1}}}
    \newcommand{\RegionMarkerTok}[1]{{#1}}
    \newcommand{\ErrorTok}[1]{\textcolor[rgb]{1.00,0.00,0.00}{\textbf{{#1}}}}
    \newcommand{\NormalTok}[1]{{#1}}

    % Additional commands for more recent versions of Pandoc
    \newcommand{\ConstantTok}[1]{\textcolor[rgb]{0.53,0.00,0.00}{{#1}}}
    \newcommand{\SpecialCharTok}[1]{\textcolor[rgb]{0.25,0.44,0.63}{{#1}}}
    \newcommand{\VerbatimStringTok}[1]{\textcolor[rgb]{0.25,0.44,0.63}{{#1}}}
    \newcommand{\SpecialStringTok}[1]{\textcolor[rgb]{0.73,0.40,0.53}{{#1}}}
    \newcommand{\ImportTok}[1]{{#1}}
    \newcommand{\DocumentationTok}[1]{\textcolor[rgb]{0.73,0.13,0.13}{\textit{{#1}}}}
    \newcommand{\AnnotationTok}[1]{\textcolor[rgb]{0.38,0.63,0.69}{\textbf{\textit{{#1}}}}}
    \newcommand{\CommentVarTok}[1]{\textcolor[rgb]{0.38,0.63,0.69}{\textbf{\textit{{#1}}}}}
    \newcommand{\VariableTok}[1]{\textcolor[rgb]{0.10,0.09,0.49}{{#1}}}
    \newcommand{\ControlFlowTok}[1]{\textcolor[rgb]{0.00,0.44,0.13}{\textbf{{#1}}}}
    \newcommand{\OperatorTok}[1]{\textcolor[rgb]{0.40,0.40,0.40}{{#1}}}
    \newcommand{\BuiltInTok}[1]{{#1}}
    \newcommand{\ExtensionTok}[1]{{#1}}
    \newcommand{\PreprocessorTok}[1]{\textcolor[rgb]{0.74,0.48,0.00}{{#1}}}
    \newcommand{\AttributeTok}[1]{\textcolor[rgb]{0.49,0.56,0.16}{{#1}}}
    \newcommand{\InformationTok}[1]{\textcolor[rgb]{0.38,0.63,0.69}{\textbf{\textit{{#1}}}}}
    \newcommand{\WarningTok}[1]{\textcolor[rgb]{0.38,0.63,0.69}{\textbf{\textit{{#1}}}}}


    % Define a nice break command that doesn't care if a line doesn't already
    % exist.
    \def\br{\hspace*{\fill} \\* }
    % Math Jax compatibility definitions
    \def\gt{>}
    \def\lt{<}
    \let\Oldtex\TeX
    \let\Oldlatex\LaTeX
    \renewcommand{\TeX}{\textrm{\Oldtex}}
    \renewcommand{\LaTeX}{\textrm{\Oldlatex}}
    % Document parameters
    % Document title
    \title{project\_01\_da\_music}
    
    
    
    
    
% Pygments definitions
\makeatletter
\def\PY@reset{\let\PY@it=\relax \let\PY@bf=\relax%
    \let\PY@ul=\relax \let\PY@tc=\relax%
    \let\PY@bc=\relax \let\PY@ff=\relax}
\def\PY@tok#1{\csname PY@tok@#1\endcsname}
\def\PY@toks#1+{\ifx\relax#1\empty\else%
    \PY@tok{#1}\expandafter\PY@toks\fi}
\def\PY@do#1{\PY@bc{\PY@tc{\PY@ul{%
    \PY@it{\PY@bf{\PY@ff{#1}}}}}}}
\def\PY#1#2{\PY@reset\PY@toks#1+\relax+\PY@do{#2}}

\@namedef{PY@tok@w}{\def\PY@tc##1{\textcolor[rgb]{0.73,0.73,0.73}{##1}}}
\@namedef{PY@tok@c}{\let\PY@it=\textit\def\PY@tc##1{\textcolor[rgb]{0.24,0.48,0.48}{##1}}}
\@namedef{PY@tok@cp}{\def\PY@tc##1{\textcolor[rgb]{0.61,0.40,0.00}{##1}}}
\@namedef{PY@tok@k}{\let\PY@bf=\textbf\def\PY@tc##1{\textcolor[rgb]{0.00,0.50,0.00}{##1}}}
\@namedef{PY@tok@kp}{\def\PY@tc##1{\textcolor[rgb]{0.00,0.50,0.00}{##1}}}
\@namedef{PY@tok@kt}{\def\PY@tc##1{\textcolor[rgb]{0.69,0.00,0.25}{##1}}}
\@namedef{PY@tok@o}{\def\PY@tc##1{\textcolor[rgb]{0.40,0.40,0.40}{##1}}}
\@namedef{PY@tok@ow}{\let\PY@bf=\textbf\def\PY@tc##1{\textcolor[rgb]{0.67,0.13,1.00}{##1}}}
\@namedef{PY@tok@nb}{\def\PY@tc##1{\textcolor[rgb]{0.00,0.50,0.00}{##1}}}
\@namedef{PY@tok@nf}{\def\PY@tc##1{\textcolor[rgb]{0.00,0.00,1.00}{##1}}}
\@namedef{PY@tok@nc}{\let\PY@bf=\textbf\def\PY@tc##1{\textcolor[rgb]{0.00,0.00,1.00}{##1}}}
\@namedef{PY@tok@nn}{\let\PY@bf=\textbf\def\PY@tc##1{\textcolor[rgb]{0.00,0.00,1.00}{##1}}}
\@namedef{PY@tok@ne}{\let\PY@bf=\textbf\def\PY@tc##1{\textcolor[rgb]{0.80,0.25,0.22}{##1}}}
\@namedef{PY@tok@nv}{\def\PY@tc##1{\textcolor[rgb]{0.10,0.09,0.49}{##1}}}
\@namedef{PY@tok@no}{\def\PY@tc##1{\textcolor[rgb]{0.53,0.00,0.00}{##1}}}
\@namedef{PY@tok@nl}{\def\PY@tc##1{\textcolor[rgb]{0.46,0.46,0.00}{##1}}}
\@namedef{PY@tok@ni}{\let\PY@bf=\textbf\def\PY@tc##1{\textcolor[rgb]{0.44,0.44,0.44}{##1}}}
\@namedef{PY@tok@na}{\def\PY@tc##1{\textcolor[rgb]{0.41,0.47,0.13}{##1}}}
\@namedef{PY@tok@nt}{\let\PY@bf=\textbf\def\PY@tc##1{\textcolor[rgb]{0.00,0.50,0.00}{##1}}}
\@namedef{PY@tok@nd}{\def\PY@tc##1{\textcolor[rgb]{0.67,0.13,1.00}{##1}}}
\@namedef{PY@tok@s}{\def\PY@tc##1{\textcolor[rgb]{0.73,0.13,0.13}{##1}}}
\@namedef{PY@tok@sd}{\let\PY@it=\textit\def\PY@tc##1{\textcolor[rgb]{0.73,0.13,0.13}{##1}}}
\@namedef{PY@tok@si}{\let\PY@bf=\textbf\def\PY@tc##1{\textcolor[rgb]{0.64,0.35,0.47}{##1}}}
\@namedef{PY@tok@se}{\let\PY@bf=\textbf\def\PY@tc##1{\textcolor[rgb]{0.67,0.36,0.12}{##1}}}
\@namedef{PY@tok@sr}{\def\PY@tc##1{\textcolor[rgb]{0.64,0.35,0.47}{##1}}}
\@namedef{PY@tok@ss}{\def\PY@tc##1{\textcolor[rgb]{0.10,0.09,0.49}{##1}}}
\@namedef{PY@tok@sx}{\def\PY@tc##1{\textcolor[rgb]{0.00,0.50,0.00}{##1}}}
\@namedef{PY@tok@m}{\def\PY@tc##1{\textcolor[rgb]{0.40,0.40,0.40}{##1}}}
\@namedef{PY@tok@gh}{\let\PY@bf=\textbf\def\PY@tc##1{\textcolor[rgb]{0.00,0.00,0.50}{##1}}}
\@namedef{PY@tok@gu}{\let\PY@bf=\textbf\def\PY@tc##1{\textcolor[rgb]{0.50,0.00,0.50}{##1}}}
\@namedef{PY@tok@gd}{\def\PY@tc##1{\textcolor[rgb]{0.63,0.00,0.00}{##1}}}
\@namedef{PY@tok@gi}{\def\PY@tc##1{\textcolor[rgb]{0.00,0.52,0.00}{##1}}}
\@namedef{PY@tok@gr}{\def\PY@tc##1{\textcolor[rgb]{0.89,0.00,0.00}{##1}}}
\@namedef{PY@tok@ge}{\let\PY@it=\textit}
\@namedef{PY@tok@gs}{\let\PY@bf=\textbf}
\@namedef{PY@tok@gp}{\let\PY@bf=\textbf\def\PY@tc##1{\textcolor[rgb]{0.00,0.00,0.50}{##1}}}
\@namedef{PY@tok@go}{\def\PY@tc##1{\textcolor[rgb]{0.44,0.44,0.44}{##1}}}
\@namedef{PY@tok@gt}{\def\PY@tc##1{\textcolor[rgb]{0.00,0.27,0.87}{##1}}}
\@namedef{PY@tok@err}{\def\PY@bc##1{{\setlength{\fboxsep}{\string -\fboxrule}\fcolorbox[rgb]{1.00,0.00,0.00}{1,1,1}{\strut ##1}}}}
\@namedef{PY@tok@kc}{\let\PY@bf=\textbf\def\PY@tc##1{\textcolor[rgb]{0.00,0.50,0.00}{##1}}}
\@namedef{PY@tok@kd}{\let\PY@bf=\textbf\def\PY@tc##1{\textcolor[rgb]{0.00,0.50,0.00}{##1}}}
\@namedef{PY@tok@kn}{\let\PY@bf=\textbf\def\PY@tc##1{\textcolor[rgb]{0.00,0.50,0.00}{##1}}}
\@namedef{PY@tok@kr}{\let\PY@bf=\textbf\def\PY@tc##1{\textcolor[rgb]{0.00,0.50,0.00}{##1}}}
\@namedef{PY@tok@bp}{\def\PY@tc##1{\textcolor[rgb]{0.00,0.50,0.00}{##1}}}
\@namedef{PY@tok@fm}{\def\PY@tc##1{\textcolor[rgb]{0.00,0.00,1.00}{##1}}}
\@namedef{PY@tok@vc}{\def\PY@tc##1{\textcolor[rgb]{0.10,0.09,0.49}{##1}}}
\@namedef{PY@tok@vg}{\def\PY@tc##1{\textcolor[rgb]{0.10,0.09,0.49}{##1}}}
\@namedef{PY@tok@vi}{\def\PY@tc##1{\textcolor[rgb]{0.10,0.09,0.49}{##1}}}
\@namedef{PY@tok@vm}{\def\PY@tc##1{\textcolor[rgb]{0.10,0.09,0.49}{##1}}}
\@namedef{PY@tok@sa}{\def\PY@tc##1{\textcolor[rgb]{0.73,0.13,0.13}{##1}}}
\@namedef{PY@tok@sb}{\def\PY@tc##1{\textcolor[rgb]{0.73,0.13,0.13}{##1}}}
\@namedef{PY@tok@sc}{\def\PY@tc##1{\textcolor[rgb]{0.73,0.13,0.13}{##1}}}
\@namedef{PY@tok@dl}{\def\PY@tc##1{\textcolor[rgb]{0.73,0.13,0.13}{##1}}}
\@namedef{PY@tok@s2}{\def\PY@tc##1{\textcolor[rgb]{0.73,0.13,0.13}{##1}}}
\@namedef{PY@tok@sh}{\def\PY@tc##1{\textcolor[rgb]{0.73,0.13,0.13}{##1}}}
\@namedef{PY@tok@s1}{\def\PY@tc##1{\textcolor[rgb]{0.73,0.13,0.13}{##1}}}
\@namedef{PY@tok@mb}{\def\PY@tc##1{\textcolor[rgb]{0.40,0.40,0.40}{##1}}}
\@namedef{PY@tok@mf}{\def\PY@tc##1{\textcolor[rgb]{0.40,0.40,0.40}{##1}}}
\@namedef{PY@tok@mh}{\def\PY@tc##1{\textcolor[rgb]{0.40,0.40,0.40}{##1}}}
\@namedef{PY@tok@mi}{\def\PY@tc##1{\textcolor[rgb]{0.40,0.40,0.40}{##1}}}
\@namedef{PY@tok@il}{\def\PY@tc##1{\textcolor[rgb]{0.40,0.40,0.40}{##1}}}
\@namedef{PY@tok@mo}{\def\PY@tc##1{\textcolor[rgb]{0.40,0.40,0.40}{##1}}}
\@namedef{PY@tok@ch}{\let\PY@it=\textit\def\PY@tc##1{\textcolor[rgb]{0.24,0.48,0.48}{##1}}}
\@namedef{PY@tok@cm}{\let\PY@it=\textit\def\PY@tc##1{\textcolor[rgb]{0.24,0.48,0.48}{##1}}}
\@namedef{PY@tok@cpf}{\let\PY@it=\textit\def\PY@tc##1{\textcolor[rgb]{0.24,0.48,0.48}{##1}}}
\@namedef{PY@tok@c1}{\let\PY@it=\textit\def\PY@tc##1{\textcolor[rgb]{0.24,0.48,0.48}{##1}}}
\@namedef{PY@tok@cs}{\let\PY@it=\textit\def\PY@tc##1{\textcolor[rgb]{0.24,0.48,0.48}{##1}}}

\def\PYZbs{\char`\\}
\def\PYZus{\char`\_}
\def\PYZob{\char`\{}
\def\PYZcb{\char`\}}
\def\PYZca{\char`\^}
\def\PYZam{\char`\&}
\def\PYZlt{\char`\<}
\def\PYZgt{\char`\>}
\def\PYZsh{\char`\#}
\def\PYZpc{\char`\%}
\def\PYZdl{\char`\$}
\def\PYZhy{\char`\-}
\def\PYZsq{\char`\'}
\def\PYZdq{\char`\"}
\def\PYZti{\char`\~}
% for compatibility with earlier versions
\def\PYZat{@}
\def\PYZlb{[}
\def\PYZrb{]}
\makeatother


    % For linebreaks inside Verbatim environment from package fancyvrb.
    \makeatletter
        \newbox\Wrappedcontinuationbox
        \newbox\Wrappedvisiblespacebox
        \newcommand*\Wrappedvisiblespace {\textcolor{red}{\textvisiblespace}}
        \newcommand*\Wrappedcontinuationsymbol {\textcolor{red}{\llap{\tiny$\m@th\hookrightarrow$}}}
        \newcommand*\Wrappedcontinuationindent {3ex }
        \newcommand*\Wrappedafterbreak {\kern\Wrappedcontinuationindent\copy\Wrappedcontinuationbox}
        % Take advantage of the already applied Pygments mark-up to insert
        % potential linebreaks for TeX processing.
        %        {, <, #, %, $, ' and ": go to next line.
        %        _, }, ^, &, >, - and ~: stay at end of broken line.
        % Use of \textquotesingle for straight quote.
        \newcommand*\Wrappedbreaksatspecials {%
            \def\PYGZus{\discretionary{\char`\_}{\Wrappedafterbreak}{\char`\_}}%
            \def\PYGZob{\discretionary{}{\Wrappedafterbreak\char`\{}{\char`\{}}%
            \def\PYGZcb{\discretionary{\char`\}}{\Wrappedafterbreak}{\char`\}}}%
            \def\PYGZca{\discretionary{\char`\^}{\Wrappedafterbreak}{\char`\^}}%
            \def\PYGZam{\discretionary{\char`\&}{\Wrappedafterbreak}{\char`\&}}%
            \def\PYGZlt{\discretionary{}{\Wrappedafterbreak\char`\<}{\char`\<}}%
            \def\PYGZgt{\discretionary{\char`\>}{\Wrappedafterbreak}{\char`\>}}%
            \def\PYGZsh{\discretionary{}{\Wrappedafterbreak\char`\#}{\char`\#}}%
            \def\PYGZpc{\discretionary{}{\Wrappedafterbreak\char`\%}{\char`\%}}%
            \def\PYGZdl{\discretionary{}{\Wrappedafterbreak\char`\$}{\char`\$}}%
            \def\PYGZhy{\discretionary{\char`\-}{\Wrappedafterbreak}{\char`\-}}%
            \def\PYGZsq{\discretionary{}{\Wrappedafterbreak\textquotesingle}{\textquotesingle}}%
            \def\PYGZdq{\discretionary{}{\Wrappedafterbreak\char`\"}{\char`\"}}%
            \def\PYGZti{\discretionary{\char`\~}{\Wrappedafterbreak}{\char`\~}}%
        }
        % Some characters . , ; ? ! / are not pygmentized.
        % This macro makes them "active" and they will insert potential linebreaks
        \newcommand*\Wrappedbreaksatpunct {%
            \lccode`\~`\.\lowercase{\def~}{\discretionary{\hbox{\char`\.}}{\Wrappedafterbreak}{\hbox{\char`\.}}}%
            \lccode`\~`\,\lowercase{\def~}{\discretionary{\hbox{\char`\,}}{\Wrappedafterbreak}{\hbox{\char`\,}}}%
            \lccode`\~`\;\lowercase{\def~}{\discretionary{\hbox{\char`\;}}{\Wrappedafterbreak}{\hbox{\char`\;}}}%
            \lccode`\~`\:\lowercase{\def~}{\discretionary{\hbox{\char`\:}}{\Wrappedafterbreak}{\hbox{\char`\:}}}%
            \lccode`\~`\?\lowercase{\def~}{\discretionary{\hbox{\char`\?}}{\Wrappedafterbreak}{\hbox{\char`\?}}}%
            \lccode`\~`\!\lowercase{\def~}{\discretionary{\hbox{\char`\!}}{\Wrappedafterbreak}{\hbox{\char`\!}}}%
            \lccode`\~`\/\lowercase{\def~}{\discretionary{\hbox{\char`\/}}{\Wrappedafterbreak}{\hbox{\char`\/}}}%
            \catcode`\.\active
            \catcode`\,\active
            \catcode`\;\active
            \catcode`\:\active
            \catcode`\?\active
            \catcode`\!\active
            \catcode`\/\active
            \lccode`\~`\~
        }
    \makeatother

    \let\OriginalVerbatim=\Verbatim
    \makeatletter
    \renewcommand{\Verbatim}[1][1]{%
        %\parskip\z@skip
        \sbox\Wrappedcontinuationbox {\Wrappedcontinuationsymbol}%
        \sbox\Wrappedvisiblespacebox {\FV@SetupFont\Wrappedvisiblespace}%
        \def\FancyVerbFormatLine ##1{\hsize\linewidth
            \vtop{\raggedright\hyphenpenalty\z@\exhyphenpenalty\z@
                \doublehyphendemerits\z@\finalhyphendemerits\z@
                \strut ##1\strut}%
        }%
        % If the linebreak is at a space, the latter will be displayed as visible
        % space at end of first line, and a continuation symbol starts next line.
        % Stretch/shrink are however usually zero for typewriter font.
        \def\FV@Space {%
            \nobreak\hskip\z@ plus\fontdimen3\font minus\fontdimen4\font
            \discretionary{\copy\Wrappedvisiblespacebox}{\Wrappedafterbreak}
            {\kern\fontdimen2\font}%
        }%

        % Allow breaks at special characters using \PYG... macros.
        \Wrappedbreaksatspecials
        % Breaks at punctuation characters . , ; ? ! and / need catcode=\active
        \OriginalVerbatim[#1,codes*=\Wrappedbreaksatpunct]%
    }
    \makeatother

    % Exact colors from NB
    \definecolor{incolor}{HTML}{303F9F}
    \definecolor{outcolor}{HTML}{D84315}
    \definecolor{cellborder}{HTML}{CFCFCF}
    \definecolor{cellbackground}{HTML}{F7F7F7}

    % prompt
    \makeatletter
    \newcommand{\boxspacing}{\kern\kvtcb@left@rule\kern\kvtcb@boxsep}
    \makeatother
    \newcommand{\prompt}[4]{
        {\ttfamily\llap{{\color{#2}[#3]:\hspace{3pt}#4}}\vspace{-\baselineskip}}
    }
    

    
    % Prevent overflowing lines due to hard-to-break entities
    \sloppy
    % Setup hyperref package
    \hypersetup{
      breaklinks=true,  % so long urls are correctly broken across lines
      colorlinks=true,
      urlcolor=urlcolor,
      linkcolor=linkcolor,
      citecolor=citecolor,
      }
    % Slightly bigger margins than the latex defaults
    
    \geometry{verbose,tmargin=1in,bmargin=1in,lmargin=1in,rmargin=1in}
    
    

\begin{document}
    
    \maketitle
    
    

    
    Содержание{}

{{1~~}Изучение данных}

{{1.1~~}Импортируем библиотеки}

{{1.2~~}Считываем данные из CSV-файла в датафрейм и сохраняем в
переменные}

{{1.3~~}Выводим основную информацию о датафреймах методом info}

{{2~~}Предобработка данных}

{{2.1~~}Преведём заголовки к одному стилю}

{{2.2~~}Заполним пропущенные значения}

{{2.3~~}Найдем и обработаем дубликаты в данных}

{{2.3.1~~}Обработаем явные дубликаты}

{{2.3.2~~}Обработаем неявные дубликаты}

{{3~~}Проверка гипотез}

{{3.1~~}Сравнение поведения пользователей двух столиц}

{{3.2~~}Музыка в начале и в конце недели}

{{3.3~~}Жанровые предпочтения в Москве и Петербурге}

{{4~~}Итоговый вывод}

    \hypertarget{ux43cux443ux437ux44bux43aux430-ux431ux43eux43bux44cux448ux438ux445-ux433ux43eux440ux43eux434ux43eux432}{%
\section{\texorpdfstring{\textbf{Музыка больших
городов}}{Музыка больших городов}}\label{ux43cux443ux437ux44bux43aux430-ux431ux43eux43bux44cux448ux438ux445-ux433ux43eux440ux43eux434ux43eux432}}

    \textbf{Описание проекта}

Заказчик просит провести сравнение музыкальных предпочтений у
пользователей Яндекс.Музыки, проживающих в Москве и Санкт-Петербурге.

    \textbf{Цель проекта}

Нужно проверить три гипотезы: 1. Активность пользователей зависит от дня
недели. Причём в Москве и Петербурге это проявляется по-разному. 2. В
понедельник утром в Москве преобладают одни жанры, а в Петербурге ---
другие. Так же и вечером пятницы преобладают разные жанры --- в
зависимости от города. 3. Москва и Петербург предпочитают разные жанры
музыки. В Москве чаще слушают поп-музыку, в Петербурге --- русский рэп.

    \textbf{План проекта}

\begin{enumerate}
\def\labelenumi{\arabic{enumi}.}
\tightlist
\item
  \hyperref[изучение-данных]{Изучение данных}\\
  1.1 \hyperref[импортируем-библиотеки]{Импортируем библиотеки}\\
  1.2
  \hyperref[считываем-данные-из-csv-файла-в-датафрейм-и-сохраняем-в-переменные]{Считываем данные из CSV-файла в датафрейм и сохраняем в переменные}\\
  1.3
  \hyperref[выводим-основную-информацию-о-датафреймах-методом-info]{Выводим основную информацию о датафреймах методом info}\\
\item
  \hyperref[предобработка-данных]{Предобработка данных}\\
  2.1
  \hyperref[преведём-заголовки-к-одному-стилю]{Преведём заголовки к одному стилю}\\
  2.2
  \hyperref[заполним-пропущенные-значения]{Заполним пропущенные значения}\\
  2.3
  \hyperref[найдем-и-обработаем-дубликаты-в-данных]{Найдем и обработаем дубликаты в данных}\\
  2.3.1
  \hyperref[обработаем-явные-дубликаты]{Обработаем явные дубликаты}\\
  2.3.2
  \hyperref[обработаем-неявные-дубликаты]{Обработаем неявные дубликаты}\\
\item
  \hyperref[проверка-гипотез]{Проверка гипотез}\\
  3.1
  \hyperref[сравнение-поведения-пользователей-двух-столиц]{Сравнение поведения пользователей двух столиц}\\
  3.2
  \hyperref[музыка-в-начале-и-в-конце-недели]{Музыка в начале и в конце недели}\\
  3.3
  \hyperref[жанровые-предпочтения-в-москве-и-петербурге]{Жанровые предпочтения в Москве и Петербурге}\\
\item
  \hyperref[итоговый-вывод]{Итоговый вывод}
\end{enumerate}

    \hypertarget{ux438ux437ux443ux447ux435ux43dux438ux435-ux434ux430ux43dux43dux44bux445}{%
\subsection{Изучение
данных}\label{ux438ux437ux443ux447ux435ux43dux438ux435-ux434ux430ux43dux43dux44bux445}}

    \hypertarget{ux438ux43cux43fux43eux440ux442ux438ux440ux443ux435ux43c-ux431ux438ux431ux43bux438ux43eux442ux435ux43aux438}{%
\subsubsection{Импортируем
библиотеки}\label{ux438ux43cux43fux43eux440ux442ux438ux440ux443ux435ux43c-ux431ux438ux431ux43bux438ux43eux442ux435ux43aux438}}

    \begin{tcolorbox}[breakable, size=fbox, boxrule=1pt, pad at break*=1mm,colback=cellbackground, colframe=cellborder]
\prompt{In}{incolor}{1}{\boxspacing}
\begin{Verbatim}[commandchars=\\\{\}]
\PY{c+c1}{\PYZsh{} импорт библиотеки pandas}
\PY{k+kn}{import} \PY{n+nn}{pandas} \PY{k}{as} \PY{n+nn}{pd} 
\end{Verbatim}
\end{tcolorbox}

    \hypertarget{ux441ux447ux438ux442ux44bux432ux430ux435ux43c-ux434ux430ux43dux43dux44bux435-ux438ux437-csv-ux444ux430ux439ux43bux430-ux432-ux434ux430ux442ux430ux444ux440ux435ux439ux43c-ux438-ux441ux43eux445ux440ux430ux43dux44fux435ux43c-ux432-ux43fux435ux440ux435ux43cux435ux43dux43dux44bux435}{%
\subsubsection{Считываем данные из CSV-файла в датафрейм и сохраняем в
переменные}\label{ux441ux447ux438ux442ux44bux432ux430ux435ux43c-ux434ux430ux43dux43dux44bux435-ux438ux437-csv-ux444ux430ux439ux43bux430-ux432-ux434ux430ux442ux430ux444ux440ux435ux439ux43c-ux438-ux441ux43eux445ux440ux430ux43dux44fux435ux43c-ux432-ux43fux435ux440ux435ux43cux435ux43dux43dux44bux435}}

    \begin{tcolorbox}[breakable, size=fbox, boxrule=1pt, pad at break*=1mm,colback=cellbackground, colframe=cellborder]
\prompt{In}{incolor}{2}{\boxspacing}
\begin{Verbatim}[commandchars=\\\{\}]
\PY{c+c1}{\PYZsh{} чтение файла с данными и сохранение в df}
\PY{k}{try}\PY{p}{:} 
    \PY{n}{df} \PY{o}{=} \PY{n}{pd}\PY{o}{.}\PY{n}{read\PYZus{}csv}\PY{p}{(}\PY{l+s+s1}{\PYZsq{}}\PY{l+s+s1}{d:/Data\PYZus{}science/Projects\PYZus{}jupiter/data/yandex\PYZus{}music\PYZus{}project.csv}\PY{l+s+s1}{\PYZsq{}}\PY{p}{)}
\PY{k}{except}\PY{p}{:}
    \PY{n}{df} \PY{o}{=} \PY{n}{pd}\PY{o}{.}\PY{n}{read\PYZus{}csv}\PY{p}{(}\PY{l+s+s1}{\PYZsq{}}\PY{l+s+s1}{/datasets/yandex\PYZus{}music\PYZus{}project.csv}\PY{l+s+s1}{\PYZsq{}}\PY{p}{)}
\end{Verbatim}
\end{tcolorbox}

    \begin{tcolorbox}[breakable, size=fbox, boxrule=1pt, pad at break*=1mm,colback=cellbackground, colframe=cellborder]
\prompt{In}{incolor}{3}{\boxspacing}
\begin{Verbatim}[commandchars=\\\{\}]
\PY{c+c1}{\PYZsh{} получение первых 10 строк таблицы df}
\PY{n}{df}\PY{o}{.}\PY{n}{head}\PY{p}{(}\PY{l+m+mi}{10}\PY{p}{)}
\end{Verbatim}
\end{tcolorbox}

            \begin{tcolorbox}[breakable, size=fbox, boxrule=.5pt, pad at break*=1mm, opacityfill=0]
\prompt{Out}{outcolor}{3}{\boxspacing}
\begin{Verbatim}[commandchars=\\\{\}]
     userID                        Track            artist   genre  \textbackslash{}
0  FFB692EC            Kamigata To Boots  The Mass Missile    rock
1  55204538  Delayed Because of Accident  Andreas Rönnberg    rock
2    20EC38            Funiculì funiculà       Mario Lanza     pop
3  A3DD03C9        Dragons in the Sunset        Fire + Ice    folk
4  E2DC1FAE                  Soul People        Space Echo   dance
5  842029A1                    Преданная         IMPERVTOR  rusrap
6  4CB90AA5                         True      Roman Messer   dance
7  F03E1C1F             Feeling This Way   Polina Griffith   dance
8  8FA1D3BE     И вновь продолжается бой               NaN  ruspop
9  E772D5C0                    Pessimist               NaN   dance

             City        time        Day
0  Saint-Petersburg  20:28:33  Wednesday
1            Moscow  14:07:09     Friday
2  Saint-Petersburg  20:58:07  Wednesday
3  Saint-Petersburg  08:37:09     Monday
4            Moscow  08:34:34     Monday
5  Saint-Petersburg  13:09:41     Friday
6            Moscow  13:00:07  Wednesday
7            Moscow  20:47:49  Wednesday
8            Moscow  09:17:40     Friday
9  Saint-Petersburg  21:20:49  Wednesday
\end{Verbatim}
\end{tcolorbox}
        
    \textbf{Описание данных:} * \texttt{userID} --- идентификатор
пользователя; * \texttt{Track} --- название трека;\\
* \texttt{artist} --- имя исполнителя; * \texttt{genre} --- название
жанра; * \texttt{City} --- город пользователя; * \texttt{time} --- время
начала прослушивания; * \texttt{Day} --- день недели.

    \hypertarget{ux432ux44bux432ux43eux434ux438ux43c-ux43eux441ux43dux43eux432ux43dux443ux44e-ux438ux43dux444ux43eux440ux43cux430ux446ux438ux44e-ux43e-ux434ux430ux442ux430ux444ux440ux435ux439ux43cux430ux445-ux43cux435ux442ux43eux434ux43eux43c-info}{%
\subsubsection{Выводим основную информацию о датафреймах методом
info}\label{ux432ux44bux432ux43eux434ux438ux43c-ux43eux441ux43dux43eux432ux43dux443ux44e-ux438ux43dux444ux43eux440ux43cux430ux446ux438ux44e-ux43e-ux434ux430ux442ux430ux444ux440ux435ux439ux43cux430ux445-ux43cux435ux442ux43eux434ux43eux43c-info}}

    \begin{tcolorbox}[breakable, size=fbox, boxrule=1pt, pad at break*=1mm,colback=cellbackground, colframe=cellborder]
\prompt{In}{incolor}{4}{\boxspacing}
\begin{Verbatim}[commandchars=\\\{\}]
\PY{c+c1}{\PYZsh{} получение общей информации о данных в таблице df}
\PY{n}{df}\PY{o}{.}\PY{n}{info}\PY{p}{(}\PY{p}{)} 
\end{Verbatim}
\end{tcolorbox}

    \begin{Verbatim}[commandchars=\\\{\}]
<class 'pandas.core.frame.DataFrame'>
RangeIndex: 65079 entries, 0 to 65078
Data columns (total 7 columns):
 \#   Column    Non-Null Count  Dtype
---  ------    --------------  -----
 0     userID  65079 non-null  object
 1   Track     63848 non-null  object
 2   artist    57876 non-null  object
 3   genre     63881 non-null  object
 4     City    65079 non-null  object
 5   time      65079 non-null  object
 6   Day       65079 non-null  object
dtypes: object(7)
memory usage: 3.5+ MB
    \end{Verbatim}

    \textbf{Вывод:}

Итак, в таблице семь столбцов. Тип данных во всех столбцах ---
\texttt{object}.\\
Количество значений в столбцах различается. Значит, в данных есть
пропущенные значения.

Кроме этого, в названиях колонок видны нарушения стиля: * строчные буквы
сочетаются с прописными, * встречаются пробелы, * в названии столбца
\texttt{userID} нет нижнего подчеркивания между словами.

В каждой строке таблицы --- данные о прослушанном треке. Часть колонок
описывает саму композицию: название, исполнителя и жанр. Остальные
данные рассказывают о пользователе: из какого он города, когда он слушал
музыку.

    \hypertarget{ux43fux440ux435ux434ux43eux431ux440ux430ux431ux43eux442ux43aux430-ux434ux430ux43dux43dux44bux445}{%
\subsection{Предобработка
данных}\label{ux43fux440ux435ux434ux43eux431ux440ux430ux431ux43eux442ux43aux430-ux434ux430ux43dux43dux44bux445}}

    \hypertarget{ux43fux440ux435ux432ux435ux434ux451ux43c-ux437ux430ux433ux43eux43bux43eux432ux43aux438-ux43a-ux43eux434ux43dux43eux43cux443-ux441ux442ux438ux43bux44e}{%
\subsubsection{Преведём заголовки к одному
стилю}\label{ux43fux440ux435ux432ux435ux434ux451ux43c-ux437ux430ux433ux43eux43bux43eux432ux43aux438-ux43a-ux43eux434ux43dux43eux43cux443-ux441ux442ux438ux43bux44e}}

    \begin{tcolorbox}[breakable, size=fbox, boxrule=1pt, pad at break*=1mm,colback=cellbackground, colframe=cellborder]
\prompt{In}{incolor}{5}{\boxspacing}
\begin{Verbatim}[commandchars=\\\{\}]
\PY{c+c1}{\PYZsh{} перечень названий столбцов таблицы df}
\PY{n}{df}\PY{o}{.}\PY{n}{columns}
\end{Verbatim}
\end{tcolorbox}

            \begin{tcolorbox}[breakable, size=fbox, boxrule=.5pt, pad at break*=1mm, opacityfill=0]
\prompt{Out}{outcolor}{5}{\boxspacing}
\begin{Verbatim}[commandchars=\\\{\}]
Index(['  userID', 'Track', 'artist', 'genre', '  City  ', 'time', 'Day'],
dtype='object')
\end{Verbatim}
\end{tcolorbox}
        
    \begin{tcolorbox}[breakable, size=fbox, boxrule=1pt, pad at break*=1mm,colback=cellbackground, colframe=cellborder]
\prompt{In}{incolor}{6}{\boxspacing}
\begin{Verbatim}[commandchars=\\\{\}]
\PY{c+c1}{\PYZsh{} переименование столбцов}
\PY{n}{df} \PY{o}{=} \PY{n}{df}\PY{o}{.}\PY{n}{rename}\PY{p}{(}\PY{n}{columns}\PY{o}{=}\PY{p}{\PYZob{}}\PY{l+s+s1}{\PYZsq{}}\PY{l+s+s1}{  userID}\PY{l+s+s1}{\PYZsq{}}\PY{p}{:} \PY{l+s+s1}{\PYZsq{}}\PY{l+s+s1}{user\PYZus{}id}\PY{l+s+s1}{\PYZsq{}}\PY{p}{,}
                        \PY{l+s+s1}{\PYZsq{}}\PY{l+s+s1}{Track}\PY{l+s+s1}{\PYZsq{}}\PY{p}{:} \PY{l+s+s1}{\PYZsq{}}\PY{l+s+s1}{track}\PY{l+s+s1}{\PYZsq{}}\PY{p}{,}
                        \PY{l+s+s1}{\PYZsq{}}\PY{l+s+s1}{  City  }\PY{l+s+s1}{\PYZsq{}}\PY{p}{:} \PY{l+s+s1}{\PYZsq{}}\PY{l+s+s1}{city}\PY{l+s+s1}{\PYZsq{}}\PY{p}{,}
                        \PY{l+s+s1}{\PYZsq{}}\PY{l+s+s1}{Day}\PY{l+s+s1}{\PYZsq{}}\PY{p}{:} \PY{l+s+s1}{\PYZsq{}}\PY{l+s+s1}{day}\PY{l+s+s1}{\PYZsq{}}\PY{p}{\PYZcb{}}\PY{p}{)}
\PY{c+c1}{\PYZsh{} проверка результатов \PYZhy{} перечень названий столбцов}
\PY{n}{df}\PY{o}{.}\PY{n}{columns}
\end{Verbatim}
\end{tcolorbox}

            \begin{tcolorbox}[breakable, size=fbox, boxrule=.5pt, pad at break*=1mm, opacityfill=0]
\prompt{Out}{outcolor}{6}{\boxspacing}
\begin{Verbatim}[commandchars=\\\{\}]
Index(['user\_id', 'track', 'artist', 'genre', 'city', 'time', 'day'],
dtype='object')
\end{Verbatim}
\end{tcolorbox}
        
    \hypertarget{ux437ux430ux43fux43eux43bux43dux438ux43c-ux43fux440ux43eux43fux443ux449ux435ux43dux43dux44bux435-ux437ux43dux430ux447ux435ux43dux438ux44f}{%
\subsubsection{Заполним пропущенные
значения}\label{ux437ux430ux43fux43eux43bux43dux438ux43c-ux43fux440ux43eux43fux443ux449ux435ux43dux43dux44bux435-ux437ux43dux430ux447ux435ux43dux438ux44f}}

    \begin{tcolorbox}[breakable, size=fbox, boxrule=1pt, pad at break*=1mm,colback=cellbackground, colframe=cellborder]
\prompt{In}{incolor}{7}{\boxspacing}
\begin{Verbatim}[commandchars=\\\{\}]
\PY{c+c1}{\PYZsh{} подсчёт пропусков}
\PY{n}{df}\PY{o}{.}\PY{n}{isna}\PY{p}{(}\PY{p}{)}\PY{o}{.}\PY{n}{sum}\PY{p}{(}\PY{p}{)}
\end{Verbatim}
\end{tcolorbox}

            \begin{tcolorbox}[breakable, size=fbox, boxrule=.5pt, pad at break*=1mm, opacityfill=0]
\prompt{Out}{outcolor}{7}{\boxspacing}
\begin{Verbatim}[commandchars=\\\{\}]
user\_id       0
track      1231
artist     7203
genre      1198
city          0
time          0
day           0
dtype: int64
\end{Verbatim}
\end{tcolorbox}
        
    \textbf{Замечание:} Не все пропущенные значения влияют на исследование.
Так в \texttt{track} и \texttt{artist} пропуски не важны для нашей
работы. Достаточно заменить их заглушкой.

Но пропуски в \texttt{genre} могут помешать сравнению музыкальных вкусов
в Москве и Санкт-Петербурге. На практике было бы правильно установить
причину пропусков и восстановить данные. Такой возможности нет в данной
ситуации. Придётся: * заполнить и эти пропуски заглушкой; * оценить,
насколько они повредят расчётам.

    \begin{tcolorbox}[breakable, size=fbox, boxrule=1pt, pad at break*=1mm,colback=cellbackground, colframe=cellborder]
\prompt{In}{incolor}{8}{\boxspacing}
\begin{Verbatim}[commandchars=\\\{\}]
\PY{c+c1}{\PYZsh{} замена пропущенных значений на \PYZsq{}unknown\PYZsq{}.}
\PY{c+c1}{\PYZsh{} если выяснится способ заполнения пропусков в столбце \PYZdq{}genre\PYZdq{},}
\PY{c+c1}{\PYZsh{} можно просто удалить этот столбец из списка \PYZdq{}columns\PYZus{}to\PYZus{}replace\PYZdq{}}
\PY{n}{columns\PYZus{}to\PYZus{}replace} \PY{o}{=} \PY{p}{[}\PY{l+s+s1}{\PYZsq{}}\PY{l+s+s1}{track}\PY{l+s+s1}{\PYZsq{}}\PY{p}{,} \PY{l+s+s1}{\PYZsq{}}\PY{l+s+s1}{artist}\PY{l+s+s1}{\PYZsq{}}\PY{p}{,} \PY{l+s+s1}{\PYZsq{}}\PY{l+s+s1}{genre}\PY{l+s+s1}{\PYZsq{}}\PY{p}{]}
\PY{k}{for} \PY{n}{column} \PY{o+ow}{in} \PY{n}{columns\PYZus{}to\PYZus{}replace}\PY{p}{:}
    \PY{n}{df}\PY{p}{[}\PY{n}{column}\PY{p}{]} \PY{o}{=} \PY{n}{df}\PY{p}{[}\PY{n}{column}\PY{p}{]}\PY{o}{.}\PY{n}{fillna}\PY{p}{(}\PY{l+s+s1}{\PYZsq{}}\PY{l+s+s1}{unknown}\PY{l+s+s1}{\PYZsq{}}\PY{p}{)}
\PY{c+c1}{\PYZsh{} подсчёт пропусков    }
\PY{n}{df}\PY{o}{.}\PY{n}{isna}\PY{p}{(}\PY{p}{)}\PY{o}{.}\PY{n}{sum}\PY{p}{(}\PY{p}{)}
\end{Verbatim}
\end{tcolorbox}

            \begin{tcolorbox}[breakable, size=fbox, boxrule=.5pt, pad at break*=1mm, opacityfill=0]
\prompt{Out}{outcolor}{8}{\boxspacing}
\begin{Verbatim}[commandchars=\\\{\}]
user\_id    0
track      0
artist     0
genre      0
city       0
time       0
day        0
dtype: int64
\end{Verbatim}
\end{tcolorbox}
        
    \hypertarget{ux43dux430ux439ux434ux435ux43c-ux438-ux43eux431ux440ux430ux431ux43eux442ux430ux435ux43c-ux434ux443ux431ux43bux438ux43aux430ux442ux44b-ux432-ux434ux430ux43dux43dux44bux445}{%
\subsubsection{Найдем и обработаем дубликаты в
данных}\label{ux43dux430ux439ux434ux435ux43c-ux438-ux43eux431ux440ux430ux431ux43eux442ux430ux435ux43c-ux434ux443ux431ux43bux438ux43aux430ux442ux44b-ux432-ux434ux430ux43dux43dux44bux445}}

    \hypertarget{ux43eux431ux440ux430ux431ux43eux442ux430ux435ux43c-ux44fux432ux43dux44bux435-ux434ux443ux431ux43bux438ux43aux430ux442ux44b}{%
\paragraph{Обработаем явные
дубликаты}\label{ux43eux431ux440ux430ux431ux43eux442ux430ux435ux43c-ux44fux432ux43dux44bux435-ux434ux443ux431ux43bux438ux43aux430ux442ux44b}}

    \begin{tcolorbox}[breakable, size=fbox, boxrule=1pt, pad at break*=1mm,colback=cellbackground, colframe=cellborder]
\prompt{In}{incolor}{9}{\boxspacing}
\begin{Verbatim}[commandchars=\\\{\}]
\PY{c+c1}{\PYZsh{} подсчёт явных дубликатов}
\PY{n}{df}\PY{o}{.}\PY{n}{duplicated}\PY{p}{(}\PY{p}{)}\PY{o}{.}\PY{n}{sum}\PY{p}{(}\PY{p}{)}
\end{Verbatim}
\end{tcolorbox}

            \begin{tcolorbox}[breakable, size=fbox, boxrule=.5pt, pad at break*=1mm, opacityfill=0]
\prompt{Out}{outcolor}{9}{\boxspacing}
\begin{Verbatim}[commandchars=\\\{\}]
3826
\end{Verbatim}
\end{tcolorbox}
        
    \begin{tcolorbox}[breakable, size=fbox, boxrule=1pt, pad at break*=1mm,colback=cellbackground, colframe=cellborder]
\prompt{In}{incolor}{10}{\boxspacing}
\begin{Verbatim}[commandchars=\\\{\}]
\PY{c+c1}{\PYZsh{} удаление явных дубликатов}
\PY{n}{df} \PY{o}{=} \PY{n}{df}\PY{o}{.}\PY{n}{drop\PYZus{}duplicates}\PY{p}{(}\PY{p}{)}\PY{o}{.}\PY{n}{reset\PYZus{}index}\PY{p}{(}\PY{n}{drop}\PY{o}{=}\PY{k+kc}{True}\PY{p}{)}
\PY{c+c1}{\PYZsh{} проверка на отсутствие дубликатов}
\PY{n}{df}\PY{o}{.}\PY{n}{duplicated}\PY{p}{(}\PY{p}{)}\PY{o}{.}\PY{n}{sum}\PY{p}{(}\PY{p}{)}
\end{Verbatim}
\end{tcolorbox}

            \begin{tcolorbox}[breakable, size=fbox, boxrule=.5pt, pad at break*=1mm, opacityfill=0]
\prompt{Out}{outcolor}{10}{\boxspacing}
\begin{Verbatim}[commandchars=\\\{\}]
0
\end{Verbatim}
\end{tcolorbox}
        
    \hypertarget{ux43eux431ux440ux430ux431ux43eux442ux430ux435ux43c-ux43dux435ux44fux432ux43dux44bux435-ux434ux443ux431ux43bux438ux43aux430ux442ux44b}{%
\paragraph{Обработаем неявные
дубликаты}\label{ux43eux431ux440ux430ux431ux43eux442ux430ux435ux43c-ux43dux435ux44fux432ux43dux44bux435-ux434ux443ux431ux43bux438ux43aux430ux442ux44b}}

    Рассмотрим неявные дубликаты в колонке \texttt{genre}. Например,
название одного и того же жанра может быть записано немного по-разному.
Такие ошибки тоже повлияют на результат исследования.

    \begin{tcolorbox}[breakable, size=fbox, boxrule=1pt, pad at break*=1mm,colback=cellbackground, colframe=cellborder]
\prompt{In}{incolor}{11}{\boxspacing}
\begin{Verbatim}[commandchars=\\\{\}]
\PY{c+c1}{\PYZsh{} Просмотр количества уникальных названий жанров}
\PY{n+nb}{print}\PY{p}{(}\PY{l+s+s1}{\PYZsq{}}\PY{l+s+s1}{Количество уникальных жанров:}\PY{l+s+s1}{\PYZsq{}}\PY{p}{,} \PY{n}{pd}\PY{o}{.}\PY{n}{Series}\PY{p}{(}\PY{n}{df}\PY{p}{[}\PY{l+s+s1}{\PYZsq{}}\PY{l+s+s1}{genre}\PY{l+s+s1}{\PYZsq{}}\PY{p}{]}\PY{o}{.}\PY{n}{unique}\PY{p}{(}\PY{p}{)}\PY{p}{)}\PY{o}{.}\PY{n}{count}\PY{p}{(}\PY{p}{)}\PY{p}{)}
\PY{c+c1}{\PYZsh{} Просмотр уникальных названий жанров}
\PY{n}{df}\PY{p}{[}\PY{l+s+s1}{\PYZsq{}}\PY{l+s+s1}{genre}\PY{l+s+s1}{\PYZsq{}}\PY{p}{]}\PY{o}{.}\PY{n}{sort\PYZus{}values}\PY{p}{(}\PY{n}{ascending}\PY{o}{=}\PY{k+kc}{True}\PY{p}{)}\PY{o}{.}\PY{n}{unique}\PY{p}{(}\PY{p}{)}
\end{Verbatim}
\end{tcolorbox}

    \begin{Verbatim}[commandchars=\\\{\}]
Количество уникальных жанров: 290
    \end{Verbatim}

            \begin{tcolorbox}[breakable, size=fbox, boxrule=.5pt, pad at break*=1mm, opacityfill=0]
\prompt{Out}{outcolor}{11}{\boxspacing}
\begin{Verbatim}[commandchars=\\\{\}]
array(['acid', 'acoustic', 'action', 'adult', 'africa', 'afrikaans',
       'alternative', 'alternativepunk', 'ambient', 'americana',
       'animated', 'anime', 'arabesk', 'arabic', 'arena',
       'argentinetango', 'art', 'audiobook', 'author', 'avantgarde',
       'axé', 'baile', 'balkan', 'beats', 'bigroom', 'black', 'bluegrass',
       'blues', 'bollywood', 'bossa', 'brazilian', 'breakbeat', 'breaks',
       'broadway', 'cantautori', 'cantopop', 'canzone', 'caribbean',
       'caucasian', 'celtic', 'chamber', 'chanson', 'children', 'chill',
       'chinese', 'choral', 'christian', 'christmas', 'classical',
       'classicmetal', 'club', 'colombian', 'comedy', 'conjazz',
       'contemporary', 'country', 'cuban', 'dance', 'dancehall',
       'dancepop', 'dark', 'death', 'deep', 'deutschrock', 'deutschspr',
       'dirty', 'disco', 'dnb', 'documentary', 'downbeat', 'downtempo',
       'drum', 'dub', 'dubstep', 'eastern', 'easy', 'electronic',
       'electropop', 'emo', 'entehno', 'epicmetal', 'estrada', 'ethnic',
       'eurofolk', 'european', 'experimental', 'extrememetal', 'fado',
       'fairytail', 'film', 'fitness', 'flamenco', 'folk', 'folklore',
       'folkmetal', 'folkrock', 'folktronica', 'forró', 'frankreich',
       'französisch', 'french', 'funk', 'future', 'gangsta', 'garage',
       'german', 'ghazal', 'gitarre', 'glitch', 'gospel', 'gothic',
       'grime', 'grunge', 'gypsy', 'handsup', "hard'n'heavy", 'hardcore',
       'hardstyle', 'hardtechno', 'hip', 'hip-hop', 'hiphop',
       'historisch', 'holiday', 'hop', 'horror', 'house', 'hymn', 'idm',
       'independent', 'indian', 'indie', 'indipop', 'industrial',
       'inspirational', 'instrumental', 'international', 'irish', 'jam',
       'japanese', 'jazz', 'jewish', 'jpop', 'jungle', 'k-pop',
       'karadeniz', 'karaoke', 'kayokyoku', 'korean', 'laiko', 'latin',
       'latino', 'leftfield', 'local', 'lounge', 'loungeelectronic',
       'lovers', 'malaysian', 'mandopop', 'marschmusik', 'meditative',
       'mediterranean', 'melodic', 'metal', 'metalcore', 'mexican',
       'middle', 'minimal', 'miscellaneous', 'modern', 'mood', 'mpb',
       'muslim', 'native', 'neoklassik', 'neue', 'new', 'newage',
       'newwave', 'nu', 'nujazz', 'numetal', 'oceania', 'old', 'opera',
       'orchestral', 'other', 'piano', 'podcasts', 'pop', 'popdance',
       'popelectronic', 'popeurodance', 'poprussian', 'post',
       'posthardcore', 'postrock', 'power', 'progmetal', 'progressive',
       'psychedelic', 'punjabi', 'punk', 'quebecois', 'ragga', 'ram',
       'rancheras', 'rap', 'rave', 'reggae', 'reggaeton', 'regional',
       'relax', 'religious', 'retro', 'rhythm', 'rnb', 'rnr', 'rock',
       'rockabilly', 'rockalternative', 'rockindie', 'rockother',
       'romance', 'roots', 'ruspop', 'rusrap', 'rusrock', 'russian',
       'salsa', 'samba', 'scenic', 'schlager', 'self', 'sertanejo',
       'shanson', 'shoegazing', 'showtunes', 'singer', 'ska', 'skarock',
       'slow', 'smooth', 'soft', 'soul', 'soulful', 'sound', 'soundtrack',
       'southern', 'specialty', 'speech', 'spiritual', 'sport',
       'stonerrock', 'surf', 'swing', 'synthpop', 'synthrock',
       'sängerportrait', 'tango', 'tanzorchester', 'taraftar', 'tatar',
       'tech', 'techno', 'teen', 'thrash', 'top', 'traditional',
       'tradjazz', 'trance', 'tribal', 'trip', 'triphop', 'tropical',
       'türk', 'türkçe', 'ukrrock', 'unknown', 'urban', 'uzbek',
       'variété', 'vi', 'videogame', 'vocal', 'western', 'world',
       'worldbeat', 'ïîï', 'электроника'], dtype=object)
\end{Verbatim}
\end{tcolorbox}
        
    \textbf{Замечание:}

Мы видим следующие неявные дубликаты: * \emph{hip}, * \emph{hop}, *
\emph{hip-hop}. Заменим эти ззначения на новое - \texttt{hiphop}:

    \begin{tcolorbox}[breakable, size=fbox, boxrule=1pt, pad at break*=1mm,colback=cellbackground, colframe=cellborder]
\prompt{In}{incolor}{12}{\boxspacing}
\begin{Verbatim}[commandchars=\\\{\}]
\PY{c+c1}{\PYZsh{} Устранение неявных дубликатов}
\PY{n}{df} \PY{o}{=} \PY{n}{df}\PY{o}{.}\PY{n}{replace}\PY{p}{(}\PY{p}{[}\PY{l+s+s1}{\PYZsq{}}\PY{l+s+s1}{hip}\PY{l+s+s1}{\PYZsq{}}\PY{p}{,} \PY{l+s+s1}{\PYZsq{}}\PY{l+s+s1}{hop}\PY{l+s+s1}{\PYZsq{}}\PY{p}{,} \PY{l+s+s1}{\PYZsq{}}\PY{l+s+s1}{hip\PYZhy{}hop}\PY{l+s+s1}{\PYZsq{}}\PY{p}{]}\PY{p}{,} \PY{l+s+s1}{\PYZsq{}}\PY{l+s+s1}{hiphop}\PY{l+s+s1}{\PYZsq{}}\PY{p}{)}
\PY{c+c1}{\PYZsh{} Просмотр количества уникальных названий жанров}
\PY{n+nb}{print}\PY{p}{(}\PY{l+s+s1}{\PYZsq{}}\PY{l+s+s1}{Количество уникальных жанров:}\PY{l+s+s1}{\PYZsq{}}\PY{p}{,} \PY{n}{pd}\PY{o}{.}\PY{n}{Series}\PY{p}{(}\PY{n}{df}\PY{p}{[}\PY{l+s+s1}{\PYZsq{}}\PY{l+s+s1}{genre}\PY{l+s+s1}{\PYZsq{}}\PY{p}{]}\PY{o}{.}\PY{n}{unique}\PY{p}{(}\PY{p}{)}\PY{p}{)}\PY{o}{.}\PY{n}{count}\PY{p}{(}\PY{p}{)}\PY{p}{)}
\PY{c+c1}{\PYZsh{} Просмотр уникальных названий жанров}
\PY{n}{df}\PY{p}{[}\PY{l+s+s1}{\PYZsq{}}\PY{l+s+s1}{genre}\PY{l+s+s1}{\PYZsq{}}\PY{p}{]}\PY{o}{.}\PY{n}{sort\PYZus{}values}\PY{p}{(}\PY{p}{)}\PY{o}{.}\PY{n}{unique}\PY{p}{(}\PY{p}{)}
\end{Verbatim}
\end{tcolorbox}

    \begin{Verbatim}[commandchars=\\\{\}]
Количество уникальных жанров: 287
    \end{Verbatim}

            \begin{tcolorbox}[breakable, size=fbox, boxrule=.5pt, pad at break*=1mm, opacityfill=0]
\prompt{Out}{outcolor}{12}{\boxspacing}
\begin{Verbatim}[commandchars=\\\{\}]
array(['acid', 'acoustic', 'action', 'adult', 'africa', 'afrikaans',
       'alternative', 'alternativepunk', 'ambient', 'americana',
       'animated', 'anime', 'arabesk', 'arabic', 'arena',
       'argentinetango', 'art', 'audiobook', 'author', 'avantgarde',
       'axé', 'baile', 'balkan', 'beats', 'bigroom', 'black', 'bluegrass',
       'blues', 'bollywood', 'bossa', 'brazilian', 'breakbeat', 'breaks',
       'broadway', 'cantautori', 'cantopop', 'canzone', 'caribbean',
       'caucasian', 'celtic', 'chamber', 'chanson', 'children', 'chill',
       'chinese', 'choral', 'christian', 'christmas', 'classical',
       'classicmetal', 'club', 'colombian', 'comedy', 'conjazz',
       'contemporary', 'country', 'cuban', 'dance', 'dancehall',
       'dancepop', 'dark', 'death', 'deep', 'deutschrock', 'deutschspr',
       'dirty', 'disco', 'dnb', 'documentary', 'downbeat', 'downtempo',
       'drum', 'dub', 'dubstep', 'eastern', 'easy', 'electronic',
       'electropop', 'emo', 'entehno', 'epicmetal', 'estrada', 'ethnic',
       'eurofolk', 'european', 'experimental', 'extrememetal', 'fado',
       'fairytail', 'film', 'fitness', 'flamenco', 'folk', 'folklore',
       'folkmetal', 'folkrock', 'folktronica', 'forró', 'frankreich',
       'französisch', 'french', 'funk', 'future', 'gangsta', 'garage',
       'german', 'ghazal', 'gitarre', 'glitch', 'gospel', 'gothic',
       'grime', 'grunge', 'gypsy', 'handsup', "hard'n'heavy", 'hardcore',
       'hardstyle', 'hardtechno', 'hiphop', 'historisch', 'holiday',
       'horror', 'house', 'hymn', 'idm', 'independent', 'indian', 'indie',
       'indipop', 'industrial', 'inspirational', 'instrumental',
       'international', 'irish', 'jam', 'japanese', 'jazz', 'jewish',
       'jpop', 'jungle', 'k-pop', 'karadeniz', 'karaoke', 'kayokyoku',
       'korean', 'laiko', 'latin', 'latino', 'leftfield', 'local',
       'lounge', 'loungeelectronic', 'lovers', 'malaysian', 'mandopop',
       'marschmusik', 'meditative', 'mediterranean', 'melodic', 'metal',
       'metalcore', 'mexican', 'middle', 'minimal', 'miscellaneous',
       'modern', 'mood', 'mpb', 'muslim', 'native', 'neoklassik', 'neue',
       'new', 'newage', 'newwave', 'nu', 'nujazz', 'numetal', 'oceania',
       'old', 'opera', 'orchestral', 'other', 'piano', 'podcasts', 'pop',
       'popdance', 'popelectronic', 'popeurodance', 'poprussian', 'post',
       'posthardcore', 'postrock', 'power', 'progmetal', 'progressive',
       'psychedelic', 'punjabi', 'punk', 'quebecois', 'ragga', 'ram',
       'rancheras', 'rap', 'rave', 'reggae', 'reggaeton', 'regional',
       'relax', 'religious', 'retro', 'rhythm', 'rnb', 'rnr', 'rock',
       'rockabilly', 'rockalternative', 'rockindie', 'rockother',
       'romance', 'roots', 'ruspop', 'rusrap', 'rusrock', 'russian',
       'salsa', 'samba', 'scenic', 'schlager', 'self', 'sertanejo',
       'shanson', 'shoegazing', 'showtunes', 'singer', 'ska', 'skarock',
       'slow', 'smooth', 'soft', 'soul', 'soulful', 'sound', 'soundtrack',
       'southern', 'specialty', 'speech', 'spiritual', 'sport',
       'stonerrock', 'surf', 'swing', 'synthpop', 'synthrock',
       'sängerportrait', 'tango', 'tanzorchester', 'taraftar', 'tatar',
       'tech', 'techno', 'teen', 'thrash', 'top', 'traditional',
       'tradjazz', 'trance', 'tribal', 'trip', 'triphop', 'tropical',
       'türk', 'türkçe', 'ukrrock', 'unknown', 'urban', 'uzbek',
       'variété', 'vi', 'videogame', 'vocal', 'western', 'world',
       'worldbeat', 'ïîï', 'электроника'], dtype=object)
\end{Verbatim}
\end{tcolorbox}
        
    \textbf{Вывод:}

Мы исправили заголовки, чтобы упростить работу с таблицей. Без
дубликатов исследование станет более точным. Пропущенные значения мы
заменили на \texttt{\textquotesingle{}unknown\textquotesingle{}}.\\
Ещё предстоит увидеть, не повредят ли исследованию пропуски в колонке
\texttt{genre}.

    \hypertarget{ux43fux440ux43eux432ux435ux440ux43aux430-ux433ux438ux43fux43eux442ux435ux437}{%
\subsection{Проверка
гипотез}\label{ux43fux440ux43eux432ux435ux440ux43aux430-ux433ux438ux43fux43eux442ux435ux437}}

    \hypertarget{ux441ux440ux430ux432ux43dux435ux43dux438ux435-ux43fux43eux432ux435ux434ux435ux43dux438ux44f-ux43fux43eux43bux44cux437ux43eux432ux430ux442ux435ux43bux435ux439-ux434ux432ux443ux445-ux441ux442ux43eux43bux438ux446}{%
\subsubsection{Сравнение поведения пользователей двух
столиц}\label{ux441ux440ux430ux432ux43dux435ux43dux438ux435-ux43fux43eux432ux435ux434ux435ux43dux438ux44f-ux43fux43eux43bux44cux437ux43eux432ux430ux442ux435ux43bux435ux439-ux434ux432ux443ux445-ux441ux442ux43eux43bux438ux446}}

    Первая гипотеза утверждает, что пользователи по-разному слушают музыку в
Москве и Санкт-Петербурге. Проверим это предположение по данным о трёх
днях недели --- понедельнике, среде и пятнице. Для этого:

\begin{itemize}
\tightlist
\item
  разделим пользователей Москвы и Санкт-Петербурга.
\item
  сравним, сколько треков послушала каждая группа пользователей в
  понедельник, среду и пятницу.
\end{itemize}

    \begin{tcolorbox}[breakable, size=fbox, boxrule=1pt, pad at break*=1mm,colback=cellbackground, colframe=cellborder]
\prompt{In}{incolor}{13}{\boxspacing}
\begin{Verbatim}[commandchars=\\\{\}]
\PY{c+c1}{\PYZsh{} Подсчёт прослушиваний в каждом городе}
\PY{n}{df}\PY{o}{.}\PY{n}{groupby}\PY{p}{(}\PY{l+s+s1}{\PYZsq{}}\PY{l+s+s1}{city}\PY{l+s+s1}{\PYZsq{}}\PY{p}{)}\PY{p}{[}\PY{l+s+s1}{\PYZsq{}}\PY{l+s+s1}{city}\PY{l+s+s1}{\PYZsq{}}\PY{p}{]}\PY{o}{.}\PY{n}{count}\PY{p}{(}\PY{p}{)}
\end{Verbatim}
\end{tcolorbox}

            \begin{tcolorbox}[breakable, size=fbox, boxrule=.5pt, pad at break*=1mm, opacityfill=0]
\prompt{Out}{outcolor}{13}{\boxspacing}
\begin{Verbatim}[commandchars=\\\{\}]
city
Moscow              42741
Saint-Petersburg    18512
Name: city, dtype: int64
\end{Verbatim}
\end{tcolorbox}
        
    \textbf{Замечание:} В Москве прослушиваний больше, чем в Петербурге. Из
этого не следует, что московские пользователи чаще слушают музыку.
Просто самих пользователей в Москве больше.

    \begin{tcolorbox}[breakable, size=fbox, boxrule=1pt, pad at break*=1mm,colback=cellbackground, colframe=cellborder]
\prompt{In}{incolor}{14}{\boxspacing}
\begin{Verbatim}[commandchars=\\\{\}]
\PY{c+c1}{\PYZsh{} Подсчёт прослушиваний в каждый из трёх дней}
\PY{n}{df}\PY{o}{.}\PY{n}{groupby}\PY{p}{(}\PY{l+s+s1}{\PYZsq{}}\PY{l+s+s1}{day}\PY{l+s+s1}{\PYZsq{}}\PY{p}{)}\PY{p}{[}\PY{l+s+s1}{\PYZsq{}}\PY{l+s+s1}{day}\PY{l+s+s1}{\PYZsq{}}\PY{p}{]}\PY{o}{.}\PY{n}{count}\PY{p}{(}\PY{p}{)}
\end{Verbatim}
\end{tcolorbox}

            \begin{tcolorbox}[breakable, size=fbox, boxrule=.5pt, pad at break*=1mm, opacityfill=0]
\prompt{Out}{outcolor}{14}{\boxspacing}
\begin{Verbatim}[commandchars=\\\{\}]
day
Friday       21840
Monday       21354
Wednesday    18059
Name: day, dtype: int64
\end{Verbatim}
\end{tcolorbox}
        
    \textbf{Замечание:} В среднем пользователи из двух городов менее активны
по средам. Но картина может измениться, если рассмотреть каждый город в
отдельности.

    \begin{tcolorbox}[breakable, size=fbox, boxrule=1pt, pad at break*=1mm,colback=cellbackground, colframe=cellborder]
\prompt{In}{incolor}{15}{\boxspacing}
\begin{Verbatim}[commandchars=\\\{\}]
\PY{c+c1}{\PYZsh{} Функция для подсчёта прослушиваний для конкретного города и дня.}
\PY{c+c1}{\PYZsh{} С помощью последовательной фильтрации с логической индексацией она }
\PY{c+c1}{\PYZsh{} сначала получит из исходной таблицы строки с нужным днём,}
\PY{c+c1}{\PYZsh{} затем из результата отфильтрует строки с нужным городом,}
\PY{c+c1}{\PYZsh{} методом count() посчитает количество значений в колонке user\PYZus{}id. }
\PY{c+c1}{\PYZsh{} Это количество функция вернёт в качестве результата}
\PY{k}{def} \PY{n+nf}{number\PYZus{}tracks}\PY{p}{(}\PY{n}{day}\PY{p}{,} \PY{n}{city}\PY{p}{)}\PY{p}{:}
    \PY{n}{track\PYZus{}list} \PY{o}{=} \PY{n}{df}\PY{p}{[}\PY{n}{df}\PY{p}{[}\PY{l+s+s1}{\PYZsq{}}\PY{l+s+s1}{day}\PY{l+s+s1}{\PYZsq{}}\PY{p}{]} \PY{o}{==} \PY{n}{day}\PY{p}{]}
    \PY{n}{track\PYZus{}list} \PY{o}{=} \PY{n}{track\PYZus{}list}\PY{p}{[}\PY{n}{track\PYZus{}list}\PY{p}{[}\PY{l+s+s1}{\PYZsq{}}\PY{l+s+s1}{city}\PY{l+s+s1}{\PYZsq{}}\PY{p}{]} \PY{o}{==} \PY{n}{city}\PY{p}{]}
    \PY{n}{track\PYZus{}list\PYZus{}count} \PY{o}{=} \PY{n}{track\PYZus{}list}\PY{p}{[}\PY{l+s+s1}{\PYZsq{}}\PY{l+s+s1}{user\PYZus{}id}\PY{l+s+s1}{\PYZsq{}}\PY{p}{]}\PY{o}{.}\PY{n}{count}\PY{p}{(}\PY{p}{)}
    \PY{k}{return} \PY{n}{track\PYZus{}list\PYZus{}count}
\end{Verbatim}
\end{tcolorbox}

    \begin{tcolorbox}[breakable, size=fbox, boxrule=1pt, pad at break*=1mm,colback=cellbackground, colframe=cellborder]
\prompt{In}{incolor}{16}{\boxspacing}
\begin{Verbatim}[commandchars=\\\{\}]
\PY{c+c1}{\PYZsh{} количество прослушиваний в Москве по понедельникам}
\PY{n}{mon\PYZus{}moscow} \PY{o}{=} \PY{n}{number\PYZus{}tracks}\PY{p}{(}\PY{l+s+s1}{\PYZsq{}}\PY{l+s+s1}{Monday}\PY{l+s+s1}{\PYZsq{}}\PY{p}{,} \PY{l+s+s1}{\PYZsq{}}\PY{l+s+s1}{Moscow}\PY{l+s+s1}{\PYZsq{}}\PY{p}{)}
\PY{c+c1}{\PYZsh{} количество прослушиваний в Санкт\PYZhy{}Петербурге по понедельникам}
\PY{n}{mon\PYZus{}spb} \PY{o}{=} \PY{n}{number\PYZus{}tracks}\PY{p}{(}\PY{l+s+s1}{\PYZsq{}}\PY{l+s+s1}{Monday}\PY{l+s+s1}{\PYZsq{}}\PY{p}{,} \PY{l+s+s1}{\PYZsq{}}\PY{l+s+s1}{Saint\PYZhy{}Petersburg}\PY{l+s+s1}{\PYZsq{}}\PY{p}{)}
\PY{c+c1}{\PYZsh{} количество прослушиваний в Москве по средам}
\PY{n}{wed\PYZus{}moscow} \PY{o}{=} \PY{n}{number\PYZus{}tracks}\PY{p}{(}\PY{l+s+s1}{\PYZsq{}}\PY{l+s+s1}{Wednesday}\PY{l+s+s1}{\PYZsq{}}\PY{p}{,} \PY{l+s+s1}{\PYZsq{}}\PY{l+s+s1}{Moscow}\PY{l+s+s1}{\PYZsq{}}\PY{p}{)}
\PY{c+c1}{\PYZsh{} количество прослушиваний в Санкт\PYZhy{}Петербурге по средам}
\PY{n}{wed\PYZus{}spb} \PY{o}{=} \PY{n}{number\PYZus{}tracks}\PY{p}{(}\PY{l+s+s1}{\PYZsq{}}\PY{l+s+s1}{Wednesday}\PY{l+s+s1}{\PYZsq{}}\PY{p}{,} \PY{l+s+s1}{\PYZsq{}}\PY{l+s+s1}{Saint\PYZhy{}Petersburg}\PY{l+s+s1}{\PYZsq{}}\PY{p}{)}
\PY{c+c1}{\PYZsh{} количество прослушиваний в Москве по пятницам}
\PY{n}{fri\PYZus{}moscow} \PY{o}{=} \PY{n}{number\PYZus{}tracks}\PY{p}{(}\PY{l+s+s1}{\PYZsq{}}\PY{l+s+s1}{Friday}\PY{l+s+s1}{\PYZsq{}}\PY{p}{,} \PY{l+s+s1}{\PYZsq{}}\PY{l+s+s1}{Moscow}\PY{l+s+s1}{\PYZsq{}}\PY{p}{)}
\PY{c+c1}{\PYZsh{} количество прослушиваний в Санкт\PYZhy{}Петербурге по пятницам}
\PY{n}{fri\PYZus{}spb} \PY{o}{=} \PY{n}{number\PYZus{}tracks}\PY{p}{(}\PY{l+s+s1}{\PYZsq{}}\PY{l+s+s1}{Friday}\PY{l+s+s1}{\PYZsq{}}\PY{p}{,} \PY{l+s+s1}{\PYZsq{}}\PY{l+s+s1}{Saint\PYZhy{}Petersburg}\PY{l+s+s1}{\PYZsq{}}\PY{p}{)}
\end{Verbatim}
\end{tcolorbox}

    \begin{tcolorbox}[breakable, size=fbox, boxrule=1pt, pad at break*=1mm,colback=cellbackground, colframe=cellborder]
\prompt{In}{incolor}{17}{\boxspacing}
\begin{Verbatim}[commandchars=\\\{\}]
\PY{c+c1}{\PYZsh{} создание таблицы с помощью конструктора \PYZdq{}pd.DataFrame\PYZdq{}}
\PY{n}{data} \PY{o}{=} \PY{p}{[}\PY{p}{[}\PY{l+s+s1}{\PYZsq{}}\PY{l+s+s1}{Moscow}\PY{l+s+s1}{\PYZsq{}}\PY{p}{,} \PY{n}{mon\PYZus{}moscow}\PY{p}{,} \PY{n}{wed\PYZus{}moscow}\PY{p}{,} \PY{n}{fri\PYZus{}moscow}\PY{p}{]}\PY{p}{,}
        \PY{p}{[}\PY{l+s+s1}{\PYZsq{}}\PY{l+s+s1}{Saint\PYZhy{}Petersburg}\PY{l+s+s1}{\PYZsq{}}\PY{p}{,}\PY{n}{mon\PYZus{}spb}\PY{p}{,} \PY{n}{wed\PYZus{}spb}\PY{p}{,} \PY{n}{fri\PYZus{}spb}\PY{p}{]}\PY{p}{]}
\PY{n}{columns} \PY{o}{=} \PY{p}{[}\PY{l+s+s1}{\PYZsq{}}\PY{l+s+s1}{city}\PY{l+s+s1}{\PYZsq{}}\PY{p}{,} \PY{l+s+s1}{\PYZsq{}}\PY{l+s+s1}{monday}\PY{l+s+s1}{\PYZsq{}}\PY{p}{,} \PY{l+s+s1}{\PYZsq{}}\PY{l+s+s1}{wednesday}\PY{l+s+s1}{\PYZsq{}}\PY{p}{,} \PY{l+s+s1}{\PYZsq{}}\PY{l+s+s1}{friday}\PY{l+s+s1}{\PYZsq{}}\PY{p}{]}
\PY{c+c1}{\PYZsh{} Таблица с результатами}
\PY{n}{info} \PY{o}{=} \PY{n}{pd}\PY{o}{.}\PY{n}{DataFrame}\PY{p}{(}\PY{n}{data}\PY{o}{=}\PY{n}{data}\PY{p}{,} \PY{n}{columns}\PY{o}{=}\PY{n}{columns}\PY{p}{)}
\PY{n}{info}
\end{Verbatim}
\end{tcolorbox}

            \begin{tcolorbox}[breakable, size=fbox, boxrule=.5pt, pad at break*=1mm, opacityfill=0]
\prompt{Out}{outcolor}{17}{\boxspacing}
\begin{Verbatim}[commandchars=\\\{\}]
               city  monday  wednesday  friday
0            Moscow   15740      11056   15945
1  Saint-Petersburg    5614       7003    5895
\end{Verbatim}
\end{tcolorbox}
        
    \textbf{Вывод:}

Данные показывают разницу поведения пользователей:\\
- В Москве пик прослушиваний приходится на понедельник и пятницу, а в
среду заметен спад. - В Петербурге, наоборот, больше слушают музыку по
средам. Активность в понедельник и пятницу здесь почти в равной мере
уступает среде.

Значит, данные говорят в пользу первой гипотезы.

    \hypertarget{ux43cux443ux437ux44bux43aux430-ux432-ux43dux430ux447ux430ux43bux435-ux438-ux432-ux43aux43eux43dux446ux435-ux43dux435ux434ux435ux43bux438}{%
\subsubsection{Музыка в начале и в конце
недели}\label{ux43cux443ux437ux44bux43aux430-ux432-ux43dux430ux447ux430ux43bux435-ux438-ux432-ux43aux43eux43dux446ux435-ux43dux435ux434ux435ux43bux438}}

    Согласно второй гипотезе, утром в понедельник в Москве преобладают одни
жанры, а в Петербурге --- другие. Так же и вечером в пятницу преобладают
разные жанры --- в зависимости от города.

    \begin{tcolorbox}[breakable, size=fbox, boxrule=1pt, pad at break*=1mm,colback=cellbackground, colframe=cellborder]
\prompt{In}{incolor}{18}{\boxspacing}
\begin{Verbatim}[commandchars=\\\{\}]
\PY{c+c1}{\PYZsh{} получение таблицы с данными по Москве}
\PY{n}{moscow\PYZus{}general} \PY{o}{=} \PY{n}{df}\PY{p}{[}\PY{n}{df}\PY{p}{[}\PY{l+s+s1}{\PYZsq{}}\PY{l+s+s1}{city}\PY{l+s+s1}{\PYZsq{}}\PY{p}{]} \PY{o}{==} \PY{l+s+s1}{\PYZsq{}}\PY{l+s+s1}{Moscow}\PY{l+s+s1}{\PYZsq{}}\PY{p}{]}
\PY{c+c1}{\PYZsh{} получение таблицы с данными по Санкт \PYZhy{} Петербургу}
\PY{n}{spb\PYZus{}general} \PY{o}{=} \PY{n}{df}\PY{p}{[}\PY{n}{df}\PY{p}{[}\PY{l+s+s1}{\PYZsq{}}\PY{l+s+s1}{city}\PY{l+s+s1}{\PYZsq{}}\PY{p}{]} \PY{o}{==} \PY{l+s+s1}{\PYZsq{}}\PY{l+s+s1}{Saint\PYZhy{}Petersburg}\PY{l+s+s1}{\PYZsq{}}\PY{p}{]}
\end{Verbatim}
\end{tcolorbox}

    \begin{tcolorbox}[breakable, size=fbox, boxrule=1pt, pad at break*=1mm,colback=cellbackground, colframe=cellborder]
\prompt{In}{incolor}{19}{\boxspacing}
\begin{Verbatim}[commandchars=\\\{\}]
\PY{c+c1}{\PYZsh{} создание функции genre\PYZus{}weekday() с параметрами table, day, time1, time2,}
\PY{c+c1}{\PYZsh{} которая возвращает информацию о самых популярных 10 жанрах в указанный день}
\PY{c+c1}{\PYZsh{} в заданное время с помощью последовательной фильтрации:}

\PY{k}{def} \PY{n+nf}{genre\PYZus{}weekday}\PY{p}{(}\PY{n}{df}\PY{p}{,} \PY{n}{day}\PY{p}{,} \PY{n}{time1}\PY{p}{,} \PY{n}{time2}\PY{p}{)}\PY{p}{:}
    \PY{n}{genre\PYZus{}df} \PY{o}{=} \PY{n}{df}\PY{p}{[}\PY{n}{df}\PY{p}{[}\PY{l+s+s1}{\PYZsq{}}\PY{l+s+s1}{day}\PY{l+s+s1}{\PYZsq{}}\PY{p}{]} \PY{o}{==} \PY{n}{day}\PY{p}{]}
    \PY{n}{genre\PYZus{}df} \PY{o}{=} \PY{n}{genre\PYZus{}df}\PY{p}{[}\PY{n}{genre\PYZus{}df}\PY{p}{[}\PY{l+s+s1}{\PYZsq{}}\PY{l+s+s1}{time}\PY{l+s+s1}{\PYZsq{}}\PY{p}{]} \PY{o}{\PYZlt{}} \PY{n}{time2}\PY{p}{]}
    \PY{n}{genre\PYZus{}df} \PY{o}{=} \PY{n}{genre\PYZus{}df}\PY{p}{[}\PY{n}{genre\PYZus{}df}\PY{p}{[}\PY{l+s+s1}{\PYZsq{}}\PY{l+s+s1}{time}\PY{l+s+s1}{\PYZsq{}}\PY{p}{]} \PY{o}{\PYZgt{}} \PY{n}{time1}\PY{p}{]}
    \PY{n}{genre\PYZus{}df\PYZus{}grouped} \PY{o}{=} \PY{n}{genre\PYZus{}df}\PY{o}{.}\PY{n}{groupby}\PY{p}{(}\PY{l+s+s1}{\PYZsq{}}\PY{l+s+s1}{genre}\PY{l+s+s1}{\PYZsq{}}\PY{p}{)}\PY{p}{[}\PY{l+s+s1}{\PYZsq{}}\PY{l+s+s1}{genre}\PY{l+s+s1}{\PYZsq{}}\PY{p}{]}\PY{o}{.}\PY{n}{count}\PY{p}{(}\PY{p}{)}
    \PY{n}{genre\PYZus{}df\PYZus{}sorted} \PY{o}{=} \PY{n}{genre\PYZus{}df\PYZus{}grouped}\PY{o}{.}\PY{n}{sort\PYZus{}values}\PY{p}{(}\PY{n}{ascending}\PY{o}{=}\PY{k+kc}{False}\PY{p}{)}
    \PY{k}{return} \PY{n}{genre\PYZus{}df\PYZus{}sorted}\PY{p}{[}\PY{p}{:}\PY{l+m+mi}{10}\PY{p}{]}
\end{Verbatim}
\end{tcolorbox}

    \textbf{Задание 25}

Cравните результаты функции \texttt{genre\_weekday()} для Москвы и
Санкт-Петербурга в понедельник утром (с 7:00 до 11:00) и в пятницу
вечером (с 17:00 до 23:00):

    \begin{tcolorbox}[breakable, size=fbox, boxrule=1pt, pad at break*=1mm,colback=cellbackground, colframe=cellborder]
\prompt{In}{incolor}{20}{\boxspacing}
\begin{Verbatim}[commandchars=\\\{\}]
\PY{c+c1}{\PYZsh{} вывод самых популярных жанров для Москвы в понедельник утром (с 7:00 до 11:00)}
\PY{n}{genre\PYZus{}weekday}\PY{p}{(}\PY{n}{moscow\PYZus{}general}\PY{p}{,} \PY{l+s+s1}{\PYZsq{}}\PY{l+s+s1}{Monday}\PY{l+s+s1}{\PYZsq{}}\PY{p}{,} \PY{l+s+s1}{\PYZsq{}}\PY{l+s+s1}{07:00}\PY{l+s+s1}{\PYZsq{}}\PY{p}{,} \PY{l+s+s1}{\PYZsq{}}\PY{l+s+s1}{11:00}\PY{l+s+s1}{\PYZsq{}}\PY{p}{)}
\end{Verbatim}
\end{tcolorbox}

            \begin{tcolorbox}[breakable, size=fbox, boxrule=.5pt, pad at break*=1mm, opacityfill=0]
\prompt{Out}{outcolor}{20}{\boxspacing}
\begin{Verbatim}[commandchars=\\\{\}]
genre
pop            781
dance          549
electronic     480
rock           474
hiphop         286
ruspop         186
world          181
rusrap         175
alternative    164
unknown        161
Name: genre, dtype: int64
\end{Verbatim}
\end{tcolorbox}
        
    \begin{tcolorbox}[breakable, size=fbox, boxrule=1pt, pad at break*=1mm,colback=cellbackground, colframe=cellborder]
\prompt{In}{incolor}{21}{\boxspacing}
\begin{Verbatim}[commandchars=\\\{\}]
\PY{c+c1}{\PYZsh{} вывод самых популярных жанров для Санкт\PYZhy{}Петербурга в понедельник утром (с 7:00 до 11:00)}
\PY{n}{genre\PYZus{}weekday}\PY{p}{(}\PY{n}{spb\PYZus{}general}\PY{p}{,} \PY{l+s+s1}{\PYZsq{}}\PY{l+s+s1}{Monday}\PY{l+s+s1}{\PYZsq{}}\PY{p}{,} \PY{l+s+s1}{\PYZsq{}}\PY{l+s+s1}{07:00}\PY{l+s+s1}{\PYZsq{}}\PY{p}{,} \PY{l+s+s1}{\PYZsq{}}\PY{l+s+s1}{11:00}\PY{l+s+s1}{\PYZsq{}}\PY{p}{)}
\end{Verbatim}
\end{tcolorbox}

            \begin{tcolorbox}[breakable, size=fbox, boxrule=.5pt, pad at break*=1mm, opacityfill=0]
\prompt{Out}{outcolor}{21}{\boxspacing}
\begin{Verbatim}[commandchars=\\\{\}]
genre
pop            218
dance          182
rock           162
electronic     147
hiphop          80
ruspop          64
alternative     58
rusrap          55
jazz            44
classical       40
Name: genre, dtype: int64
\end{Verbatim}
\end{tcolorbox}
        
    \begin{tcolorbox}[breakable, size=fbox, boxrule=1pt, pad at break*=1mm,colback=cellbackground, colframe=cellborder]
\prompt{In}{incolor}{22}{\boxspacing}
\begin{Verbatim}[commandchars=\\\{\}]
\PY{c+c1}{\PYZsh{} вывод самых популярных жанров для Москвы в пятницу вечером (с 17:00 до 23:00)}
\PY{n}{genre\PYZus{}weekday}\PY{p}{(}\PY{n}{moscow\PYZus{}general}\PY{p}{,} \PY{l+s+s1}{\PYZsq{}}\PY{l+s+s1}{Friday}\PY{l+s+s1}{\PYZsq{}}\PY{p}{,} \PY{l+s+s1}{\PYZsq{}}\PY{l+s+s1}{17:00}\PY{l+s+s1}{\PYZsq{}}\PY{p}{,} \PY{l+s+s1}{\PYZsq{}}\PY{l+s+s1}{23:00}\PY{l+s+s1}{\PYZsq{}}\PY{p}{)}
\end{Verbatim}
\end{tcolorbox}

            \begin{tcolorbox}[breakable, size=fbox, boxrule=.5pt, pad at break*=1mm, opacityfill=0]
\prompt{Out}{outcolor}{22}{\boxspacing}
\begin{Verbatim}[commandchars=\\\{\}]
genre
pop            713
rock           517
dance          495
electronic     482
hiphop         273
world          208
ruspop         170
alternative    163
classical      163
rusrap         142
Name: genre, dtype: int64
\end{Verbatim}
\end{tcolorbox}
        
    \begin{tcolorbox}[breakable, size=fbox, boxrule=1pt, pad at break*=1mm,colback=cellbackground, colframe=cellborder]
\prompt{In}{incolor}{23}{\boxspacing}
\begin{Verbatim}[commandchars=\\\{\}]
\PY{c+c1}{\PYZsh{} вывод самых популярных жанров для Санкт\PYZhy{}Петербурга в пятницу вечером (с 17:00 до 23:00)}
\PY{n}{genre\PYZus{}weekday}\PY{p}{(}\PY{n}{spb\PYZus{}general}\PY{p}{,} \PY{l+s+s1}{\PYZsq{}}\PY{l+s+s1}{Friday}\PY{l+s+s1}{\PYZsq{}}\PY{p}{,} \PY{l+s+s1}{\PYZsq{}}\PY{l+s+s1}{17:00}\PY{l+s+s1}{\PYZsq{}}\PY{p}{,} \PY{l+s+s1}{\PYZsq{}}\PY{l+s+s1}{23:00}\PY{l+s+s1}{\PYZsq{}}\PY{p}{)}
\end{Verbatim}
\end{tcolorbox}

            \begin{tcolorbox}[breakable, size=fbox, boxrule=.5pt, pad at break*=1mm, opacityfill=0]
\prompt{Out}{outcolor}{23}{\boxspacing}
\begin{Verbatim}[commandchars=\\\{\}]
genre
pop            256
electronic     216
rock           216
dance          210
hiphop          97
alternative     63
jazz            61
classical       60
rusrap          59
world           54
Name: genre, dtype: int64
\end{Verbatim}
\end{tcolorbox}
        
    \textbf{Вывод:}

Если сравнить топ-10 жанров в понедельник утром, можно сделать такие
выводы:

\begin{enumerate}
\def\labelenumi{\arabic{enumi}.}
\item
  В Москве и Петербурге слушают похожую музыку. Единственное отличие ---
  в московский рейтинг вошёл жанр ``world'', а в петербургский --- джаз
  и классика.
\item
  В Москве пропущенных значений оказалось так много, что значение
  \texttt{\textquotesingle{}unknown\textquotesingle{}} заняло десятое
  место среди самых популярных жанров. Значит, пропущенные значения
  занимают существенную долю в данных и угрожают достоверности
  исследования.
\end{enumerate}

Вечер пятницы не меняет эту картину. Некоторые жанры поднимаются немного
выше, другие спускаются, но в целом топ-10 остаётся тем же самым.

Таким образом, вторая гипотеза подтвердилась лишь частично: *
Пользователи слушают похожую музыку в начале недели и в конце. * Разница
между Москвой и Петербургом не слишком выражена. В Москве чаще слушают
русскую популярную музыку, в Петербурге --- джаз.

Однако пропуски в данных ставят под сомнение этот результат. В Москве их
так много, что рейтинг топ-10 мог бы выглядеть иначе, если бы не
утерянные данные о жанрах.

    \hypertarget{ux436ux430ux43dux440ux43eux432ux44bux435-ux43fux440ux435ux434ux43fux43eux447ux442ux435ux43dux438ux44f-ux432-ux43cux43eux441ux43aux432ux435-ux438-ux43fux435ux442ux435ux440ux431ux443ux440ux433ux435}{%
\subsubsection{Жанровые предпочтения в Москве и
Петербурге}\label{ux436ux430ux43dux440ux43eux432ux44bux435-ux43fux440ux435ux434ux43fux43eux447ux442ux435ux43dux438ux44f-ux432-ux43cux43eux441ux43aux432ux435-ux438-ux43fux435ux442ux435ux440ux431ux443ux440ux433ux435}}

Гипотеза: Петербург --- столица рэпа, музыку этого жанра там слушают
чаще, чем в Москве. А Москва --- город контрастов, в котором, тем не
менее, преобладает поп-музыка.

    \begin{tcolorbox}[breakable, size=fbox, boxrule=1pt, pad at break*=1mm,colback=cellbackground, colframe=cellborder]
\prompt{In}{incolor}{24}{\boxspacing}
\begin{Verbatim}[commandchars=\\\{\}]
\PY{c+c1}{\PYZsh{} группировка данных по Москве по жанру}
\PY{n}{moscow\PYZus{}genres} \PY{o}{=} \PY{n}{moscow\PYZus{}general}\PY{o}{.}\PY{n}{groupby}\PY{p}{(}\PY{l+s+s1}{\PYZsq{}}\PY{l+s+s1}{genre}\PY{l+s+s1}{\PYZsq{}}\PY{p}{)}\PY{p}{[}\PY{l+s+s1}{\PYZsq{}}\PY{l+s+s1}{genre}\PY{l+s+s1}{\PYZsq{}}\PY{p}{]}\PY{o}{.}\PY{n}{count}\PY{p}{(}\PY{p}{)}\PY{o}{.}\PY{n}{sort\PYZus{}values}\PY{p}{(}\PY{n}{ascending}\PY{o}{=}\PY{k+kc}{False}\PY{p}{)}
\PY{c+c1}{\PYZsh{} просмотр первых 10 строк moscow\PYZus{}genres}
\PY{n}{moscow\PYZus{}genres}\PY{o}{.}\PY{n}{head}\PY{p}{(}\PY{l+m+mi}{10}\PY{p}{)}
\end{Verbatim}
\end{tcolorbox}

            \begin{tcolorbox}[breakable, size=fbox, boxrule=.5pt, pad at break*=1mm, opacityfill=0]
\prompt{Out}{outcolor}{24}{\boxspacing}
\begin{Verbatim}[commandchars=\\\{\}]
genre
pop            5892
dance          4435
rock           3965
electronic     3786
hiphop         2096
classical      1616
world          1432
alternative    1379
ruspop         1372
rusrap         1161
Name: genre, dtype: int64
\end{Verbatim}
\end{tcolorbox}
        
    \begin{tcolorbox}[breakable, size=fbox, boxrule=1pt, pad at break*=1mm,colback=cellbackground, colframe=cellborder]
\prompt{In}{incolor}{25}{\boxspacing}
\begin{Verbatim}[commandchars=\\\{\}]
\PY{c+c1}{\PYZsh{} группировка данных по Санкт \PYZhy{} Петербургу по жанру}
\PY{n}{spb\PYZus{}genres} \PY{o}{=} \PY{n}{spb\PYZus{}general}\PY{o}{.}\PY{n}{groupby}\PY{p}{(}\PY{l+s+s1}{\PYZsq{}}\PY{l+s+s1}{genre}\PY{l+s+s1}{\PYZsq{}}\PY{p}{)}\PY{p}{[}\PY{l+s+s1}{\PYZsq{}}\PY{l+s+s1}{genre}\PY{l+s+s1}{\PYZsq{}}\PY{p}{]}\PY{o}{.}\PY{n}{count}\PY{p}{(}\PY{p}{)}\PY{o}{.}\PY{n}{sort\PYZus{}values}\PY{p}{(}\PY{n}{ascending}\PY{o}{=}\PY{k+kc}{False}\PY{p}{)}
\PY{c+c1}{\PYZsh{} просмотр первых 10 строк spb\PYZus{}genres}
\PY{n}{spb\PYZus{}genres}\PY{p}{[}\PY{p}{:}\PY{l+m+mi}{10}\PY{p}{]}
\end{Verbatim}
\end{tcolorbox}

            \begin{tcolorbox}[breakable, size=fbox, boxrule=.5pt, pad at break*=1mm, opacityfill=0]
\prompt{Out}{outcolor}{25}{\boxspacing}
\begin{Verbatim}[commandchars=\\\{\}]
genre
pop            2431
dance          1932
rock           1879
electronic     1736
hiphop          960
alternative     649
classical       646
rusrap          564
ruspop          538
world           515
Name: genre, dtype: int64
\end{Verbatim}
\end{tcolorbox}
        
    \textbf{Вывод:}

Гипотеза частично подтвердилась: * Поп-музыка --- самый популярный жанр
в Москве, как и предполагала гипотеза. Более того, в топ-10 жанров
встречается близкий жанр --- русская популярная музыка. * Вопреки
ожиданиям, рэп одинаково популярен в Москве и Петербурге.

    \hypertarget{ux438ux442ux43eux433ux43eux432ux44bux439-ux432ux44bux432ux43eux434}{%
\subsection{Итоговый
вывод}\label{ux438ux442ux43eux433ux43eux432ux44bux439-ux432ux44bux432ux43eux434}}

    В данном проекте данных Яндекс Музыки мы проверяли гипотезы и сравнивали
поведение пользователей двух столиц.

\textbf{Этапы выполнения проекта:}\\
1. Изучение данных.\\
Здесь мы установили и импортировали необходимые для работы библиотеки,
загрузили и изучили данные.

\begin{enumerate}
\def\labelenumi{\arabic{enumi}.}
\setcounter{enumi}{1}
\item
  Предобработка данных.\\
  На данном этапе мы привели заголовки к одному типу, заполнили пропуски
  и обработали дубликаты (явные и неявные).
\item
  Проверка гипотез.
\end{enumerate}

\textbf{В итоге:}\\
Мы проверили три гипотезы и установили: 1. День недели по-разному влияет
на активность пользователей в Москве и Петербурге.\\
Первая гипотеза полностью подтвердилась.

\begin{enumerate}
\def\labelenumi{\arabic{enumi}.}
\setcounter{enumi}{1}
\item
  Музыкальные предпочтения не сильно меняются в течение недели --- будь
  то Москва или Петербург. Небольшие различия заметны в начале недели,
  по понедельникам: в Москве слушают музыку жанра \texttt{world}, а в
  Петербурге --- джаз и классику.\\
  Таким образом, вторая гипотеза подтвердилась лишь отчасти. Этот
  результат мог оказаться иным, если бы не пропуски в данных.
\item
  Во вкусах пользователей Москвы и Петербурга больше общего чем
  различий. Вопреки ожиданиям, предпочтения жанров в Петербурге
  напоминают московские.\\
  Третья гипотеза не подтвердилась. Если различия в предпочтениях и
  существуют, на основной массе пользователей они незаметны.
\end{enumerate}


    % Add a bibliography block to the postdoc
    
    
    
\end{document}
